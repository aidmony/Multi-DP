\NeedsTeXFormat{LaTeX2e}
\documentclass[11pt]{article}
\usepackage{url}
\usepackage{amsmath}
\usepackage{amsthm}
\usepackage{amssymb}
\usepackage{mathpartir}
\usepackage{graphicx}
\usepackage{comment}
\usepackage{caption}
\usepackage{subcaption}


%commands for PL

\newcommand{\deftech}[1]{\textbf{#1}}
\newcommand\mrel{\mathop{\mathbf{r}}}
\newcommand\morel{\mathop{\mathbf{r}'}}
\newcommand\menv{\rho}
\newcommand\mval{v}
\newcommand\mans{a}
\newcommand\mint{i}
\newcommand\moint{j}
\newcommand\mbool{b}

\newcommand\plug[2]{#1[#2]}

\newcommand\merr{r}

\newcommand\mctx{\mathcal{C}}
\newcommand\mectx{\mathcal{E}}
\newtheorem{theorem}{Theorem}
\newtheorem{lemma}{Lemma}
\newtheorem{definition}{Definition}
\newcommand\Err{\mathit{Err}}
\newcommand\Plus{\mathit{Plus}}
\newcommand\Mult{\mathit{Mult}}
\newcommand\Succ{\mathit{Succ}}
\newcommand\Pred{\mathit{Pred}}
\newcommand\Eq{\mathit{Eq}}
\newcommand\True{\mathit{True}}
\newcommand\False{\mathit{False}}
\newcommand\If{\mathit{If}}
\newcommand\Div{\mathit{Div}}

\newcommand\reduce{\mathop{\mathbf{b}}}

\newcommand\areducename{\mathbf{a}}
\newcommand\areduce[2]{#1\;\areducename\;#2}

\newcommand\step{\rightarrow_\mathbf{b}}
\newcommand\multistep{\rightarrow^\star_\mathbf{b}}

\newcommand\astdstep{\longmapsto_{\reduce}}
\newcommand\astdmultistep{\longmapsto^\star_{\reduce}}

\newcommand\breducename{\mathbf{b}}
\newcommand\errreducename{\mathbf{err}}
\newcommand\propreducename{\mathbf{prop}}
\newcommand\bvreducename{\mathbf{bv}}
\newcommand\bstepname{\rightarrow_{\breducename}}
\newcommand\bmultistepname{\rightarrow^\star_{\breducename}}

\newcommand\bmultistep[3]{#1\vdash #2\;\bmultistepname\;#3}
\newcommand\bstdstep[3]{#1\vdash #2\;{\longmapsto_{\breducename}}\;#3}
\newcommand\bstdmultistep[3]{#1\vdash #2\;{\longmapsto^\star_{\breducename}}\;#3}

\newcommand\breduce[3]{#1 \vdash {#2}\;\breducename\; {#3}}
\newcommand\errreduce[3]{#1 \vdash {#2}\;\errreducename\; {#3}}
\newcommand\propreduce[3]{#1 \vdash {#2}\;\propreducename\; {#3}}
\newcommand\bvreduce[3]{#1 \vdash {#2}\;\bvreducename\; {#3}}
\newcommand\bstep[3]{#1 \vdash {#2}\;\rightarrow_{\breducename}\; {#3}}
\newcommand\bclosedstep[2]{{#1}\;\rightarrow_{\breducename}\; {#2}}

\newcommand\laxparstep{\rightrightarrows_\mathbf{a}}
\newcommand\maxparstep{\rightrightarrows'_\mathbf{a}}


\newcommand\Arith{\mathcal{A}}
\newcommand\Barith{\mathcal{B}}

\newcommand\Var{\mathit{Var}}

\newcommand{\mvar}{x}
\newcommand\s[1]{\mathit{#1}}

%commands for CW-Privacy

\newcommand{\xdown}[1]{X_{#1 \downarrow}}
\newcommand{\xup}[1]{X_{#1 \uparrow}}
\newcommand{\xdownk}[2]{X_{#1 \downarrow #2}}
\newcommand{\xupk}[2]{X_{#1 \uparrow #2}}
\newcommand{\xblock}[1]{[x_{#1 1}, x_{#1 2} , \dots, x_{#1 \frac{1}{c}}]^{\top}}
\newcommand{\conpr}[2]{Pr[#1\,|\,#2]}
\newcommand{\po}[2]{P^{0}(y_{1}, \dots , y_{#1} \,|\, #2)}
\newcommand{\pos}[2]{P^{0}(y_{#1} \,|\, #2)}

\newcommand{\pl}[2]{P^{1}(y_{1}, \dots , y_{#1} \,|\, #2)}
\newcommand{\pls}[2]{P^{1}(y_{#1} \,|\, #2)}

\newcommand{\priv}{{\bf priv}(X)}
\newcommand{\alt}[1]{{\bf alt}(X_{#1})}
\newcommand{\xbot}[1]{x_{#1 \frac{1}{c}}}
\newcommand{\cwp}[1]{(\epsilon, \delta, \Delta_{#1}, \Gamma)}
\newcommand{\xmid}[1]{X_{#1 \updownarrow}}
\newcommand{\xm}[1]{x_{#1 \circ}}



\title{Multi-party DP Write-up 2\\
Stochastic Gradient Descent}
\date{}
\begin{document}
\maketitle

\section{Notations and Settings}
Let $D={\bf [X, y]}$ be a database of size $n$, with ${\bf X} \in \mathbb{R}^{n \times p}$ and ${\bf y} \in \mathbb{R}$. $[n]$ stands for the set $\{1, \dots, n\}$. We use $z \sim N(\mu, \sigma)$ to denote a variable following the Gaussian distribution with mean $\mu$ and variance $\sigma^{2}$. 

\begin{definition}[ERM]
Given a data set $D = \{d_{1}, \dots , d_{n} \}$ drawn from a universe ${\cal X} \subseteq \mathbb{R}^{n \times (p+1)} $ , and a closed, convex set $\cal C$, the goal of Empirical Risk Minimization (ERM) is 
\begin{equation}
\min_{\theta \in {\cal C}} L(\theta; D) = \Sigma_{i=1}^{n} l(\theta; d_{i})	
\end{equation}
The function $l$ defines, for each data point $d$, a loss function $l(.; d)$ on $\cal C$. We will generally assume that $l(�; d)$ is convex and L-Lipschitz for all $d \in {\cal X}$.
\end{definition}

\begin{definition}[Lipschitz Function]
$l : {\cal C} \rightarrow \mathbb{R}$ is L-Lipschitz (in the Euclidean norm) if, for all pairs $x, y \in {\cal C}$, we have $|l(x)-l(y)| \leq L|x-y|_{2}$.
\end{definition}

\begin{definition}[Distributed Private ERM]
Let $A_{1}, \dots A_{t}$ and $B$ be $t+1$ parties. Each $A_{i}$ owns some of the data $\{d_{j}\}_{j \in S_{i}}$, where $S_{i} \subseteq [n]$ and $|S_{i}| = n_{i}$. For simplicity, we assume $\{S_{1}, \dots , S_{t}\}$ partition $[n]$. \\
Our goal is to find an algorithm that makes $\{A_{1}, \dots A_{t}, B\}$ distributively compute $\theta$ under ERM model. such that the entire computation, including all the intermediate outputs and final output, is $(\epsilon, \delta)$-differential private.
\end{definition}


{\bf Privacy $\&$ Utility.} There is always a trade-off between privacy and utility. We use the concept of worst case (over input data) {\it expected excess risk} to evaluate the utility of our algorithm.
 
\begin{definition}
Given loss function $L(\theta; D)$ over database $D$, its worst case {\it expected excess risk} is defined as
\begin{equation}
E[L(\theta; D) - L(\theta^{*}; D)]
\end{equation}
where $\theta$ is the output of the algorithm, $\theta^{*}$ is the optimal minimizer of the loss function and the randomness comes from the algorithm.
\end{definition}

\section{Stochastic Gradient Descent}
We propose two distributed algorithms, both of which are based on the stochastic gradient descent algorithm in [1].

{\bf Algorithm SGD1}.
\begin{enumerate}
\item $B$ uniformly chooses the initial assignment $\theta_{0} \in {\cal C}$ , learning rate function $\eta: [n^{2}] \rightarrow \mathbb{R}$, a noise variance $\sigma^{2} >0$, and a sequence of ${\pi}=[\pi_{1}, \dots \pi_{n^{2}}]$ with each $\pi_{j}$ uniformly chosen from $[n]$. $B$ broadcasts $(\theta_{0}, \eta, \sigma^{2})$.
\item In each iteration $i=1$ to $n^{2}$, $B$ gives a computing permission to agent $A_{j}$ who owns data $d_{\pi(i)}$.
\item Upon receiving permission, $A_{j}$ will compute $\theta_{i+1} =\Pi_{\cal C}(\theta_{i} - \eta(i) G_{i})$, where $G_{i} = n \Delta(l(\theta_{i};d_{\pi(i)})) + z_{i}$ with $z_{i} \sim N(0, \sigma^{2})^{p}$ , and $\Pi_{\cal C}$ is the projection function onto set $\cal C$. $A_{j}$ sends $G_{i}$ to $B$.
\item $B$ broadcast $G_{i}$ together with a update permission to all other $A_{k}$, where $k \neq j$.
\item Each of the remaining $A_{k}$ will update $\theta_{i+1} =\Pi_{\cal C}[ \theta_{i} - \eta(i) G_{i}] $
\item After $n^{2}$ iterations, $B$ returns $\theta = \theta_{n^{2}}$ as the final output.
\end{enumerate}
For $0< \epsilon <1$ and $0<\delta < 1/n$, we set parameters in Algorithm SGD1 as the follows
\begin{equation}
\sigma^{2} = O (\frac{L^{2} n^{2} \log (n/\delta) \log (1/\delta)}{\epsilon^{2}})
\end{equation}
\begin{equation}
\eta(t) = \frac{|{\cal C}|_{2}}{\sqrt{t(n^{2}L^{2} + p \sigma^{2})}}
\end{equation}

\begin{theorem}
Algorithm SGD1 is $(\epsilon, 1-(1-\delta/2)^{n^{2}})(1-\delta^{1/3}))$-Private. In particular, Algorithm SGD1 is $(\epsilon, n^{2}\delta/2)$ when $\delta = \omega(1/n^{3})$,
\end{theorem}
{\bf Proof. } The entire output of Algorithm SGD1 consists of $\{ G_{1}, \dots, G_{n^{2}}\}$. We show adaptively outputting $n^{2}$ such $G_{i}$ is $(\epsilon, (n^{2}+1)\delta)$-Private.

Let $G_{i}(D) = n \Delta(l(\theta_{i};d_{\pi(i)})) + z_{i}$ be a random variable defined over the randomness of $\pi$, $z_{i}$ and conditioned on $\theta_{i}$. Fixing the randomness of $\pi$, it can be shown according to standard Gaussian Differential Privacy analysis that each $G_{i}$ is $(\frac{\epsilon}{2\sqrt{\log(1/\delta)}}, \delta/2)$-DP [1]. (We just need to show with probability at least $1-\delta/2$, $|\log (\frac{G_{i}(D)}{G_{i}(D')})| \leq \frac{\epsilon}{2\sqrt{\log(1/\delta)}}$, where $D$, $D'$ are a pair of neighboring databases)

It is obvious that $\pi$ provides the same randomness as the uniform sampling from $D$. By the privacy amplification lemma in [1], each $G_{i}$ is $(\frac{\epsilon}{n\sqrt{\log(1/\delta)}}, \delta/2)$-DP. We apply the strong composition theorem lemma [2] with $k=n^{2}$, $\delta' = \delta^{1/3}$. We have
\begin{equation}
n^{2}\epsilon^{2} + \epsilon \sqrt{2 n^{2} \log(1/\delta')} = \epsilon (\frac{\epsilon}{\log(1/\delta)} + \sqrt{2/3}) \leq \epsilon
\end{equation}


\begin{lemma} [Strong Composition Theorem]
For any $1>\epsilon > 0$, $\delta \in [0,1]$ and $\delta' \in [0,1]$ the class of $(\epsilon, \delta)$-differentially private mechanisms satisfies $(\epsilon', 1-(1-\delta)^{k})(1-\delta')$-DP under $k$-fold adaptive composition for
\begin{equation}
\epsilon' = \min \{ k\epsilon, k\epsilon^{2} + \epsilon \sqrt{2k \log(1/\delta')} \}
\end{equation}
\end{lemma}

{\bf Algorithm SGD2}.
\begin{enumerate}
\item $B$ uniformly chooses the initial assignment $\theta_{0} \in {\cal C}$ , learning rate function $\eta: [n^{2}] \rightarrow \mathbb{R}$, a noise variance $\sigma^{2} >0$, and a sequence of ${\pi}=[\pi_{1}, \dots \pi_{t}]$ with each $\pi_{j}$ uniformly chosen from $[t]$. $B$ broadcasts $(\theta_{0}, \eta, \sigma^{2})$.
\item In each iteration $i=1$ to $t$, $B$ gives a computing permission to agent $A_{\pi_{i}}$.
\item Upon receiving permission, for $j=1$ to $n_{\pi_{i}}$, $A_{\pi_{i}}$ uniformly pick $d_{j}$ from his dataset $S_{\pi_{i}}$.
Then, $A_{\pi_{i}}$ will compute $\theta_{j+1} =\Pi_{\cal C}(\theta_{j} - \eta(j) G_{j})$, where $G_{j} = n \Delta(l(\theta_{j};d_{j})) + z_{j}$ with $z_{j} \sim N(0, \sigma^{2})^{p}$ , and $\Pi_{\cal C}$ is the projection function onto set $\cal C$. $A_{\pi_{i}}$ sends $\{G_{j}\}_{j\in [n_{\pi_{i}}]}$ to $B$.
\item $B$ broadcast $\{G_{j}\}_{j\in [n_{\pi_{i}}]}$ together with a update permission to all other $A_{k}$, where $k \neq j$.
\item Each of the remaining $A_{k}$ will update $n_{\pi_{i}}$ steps by $\theta_{j+1} =\Pi_{\cal C}[ \theta_{j} - \eta(j) G_{j}] $
\item After $n^{2}$ iterations, $B$ returns $\theta = \theta_{n^{2}}$ as the final output.
\end{enumerate}


\begin{theorem}
Algorithm SGD2 is 
\end{theorem}


\section{References}
\begin{enumerate}
\item http://www.cse.psu.edu/~ads22/pubs/BST14/2014-04-10-BST14-convex-opt.pdf 
\item http://arxiv.org/pdf/1311.0776v2.pdf
\end{enumerate}


\end{document}
